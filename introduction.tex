\documentclass[main.tex]{subfiles}

\begin{document}
\listoftodos

\section{Introduction}

\begin{itemize}
\item Introduction
  \begin{itemize}
  \item Motivation: Retail pricing, deplete stock, maximise revenue
  \item References: Revenue management book. Bertsimas demand-learning
  \item Many heuristics or approximations, investigate how well they
    do
  \item Won't go into computational costs
  \item Risk-aversion. Expected utility
  \end{itemize}
\item Problem formulation and optimal control
  \begin{itemize}
  \item Plot distribution of profits (next to pricing evolution)
  \end{itemize}
\item Approximations
  \begin{itemize}
  \item Partially observed system becomes intractable
    quickly (see Bertsimas)
  \item The practical setting, many dimensions to take into account
    and model
  \item Introduce lookahead: certainty equivalent and extension
  \item Theoretical results from Bertsekas, do they fit with these methods?
  \item Comparison of profits
  \item Comparison of price trajectories?
  \end{itemize}
\item Model miss-specification
  \begin{itemize}
  \item Model won't be 100\% correct. How stable are these methods to miss-specification
  \item Compare CEC with Bellman: mean, variance, distribution
    miss-specification
  \item Compare lookahead with multiple scenarios when we have mean miss-specification
  \item (Shall we argue for Bellman robustness as with parabolic equations?)
  \end{itemize}
\item Risk-aversion
  \begin{itemize}
  \item Motivation: product manager judged on profit over whole
    period. Company judged on quarterly/yearly profits/revenues, below worse
    than above expected?
  \item Reformulate with revenue as state
  \item CARA:~revenue does not make a difference. Show
  \item CRRA:~increasing price with increasing revenue. Run
    experiments with $T=2$ or $T=3$
  \item Quadratic:~opposite behaviour. Just mention
  \item These are all expected due to (C/D/I)ARA
  \end{itemize}
\item Conclusion
  \begin{itemize}
  \item Approximations used in industry. Understanding how well they
    perform is important
  \item Forward: more products. Time dependence. Constraints
  \end{itemize}
\end{itemize}

The process of pricing products in order to control demand and
maximize revenues has been undertaken for centuries. In recent
decades, data- and model-driven approaches have become increasingly
popular in order to advise on and automate the process for companies.
There are several success stories from early adopters, for example in
the airline industry.\todo{Reference}
For retailers with billions of pounds in revenues, small
improvements to their revenue-management processes have values worth millions.
Unsold items add up to thousands of tonnes of waste per year, so
a better control of the demand for products is advantageous
for both retailers and the environment.

We are interested in strategies to dynamically set the prices of products, and
formulate the problem as a stochastic optimal control problem.
Product stock levels $S_t$ are controlled by a pricing process
$\alpha_t$, and evolve according to a dynamic system with disturbances
$W_t$:
\begin{equation}
S_{t+1}=f_{t+1}(S_t,\alpha_t,W_{t+1})
\end{equation}
The goal is to find $\alpha$ which maximises total revenues over a
given time horizon $T$, and
minimises the costs of unsold stock:
\begin{equation}
  \max_{\alpha}\mathbb E_W\left[ \sum_{t=0}^TU_t(S_t,\alpha_t,W_{t+1})
  + \overline{U}(S_T)\right]
\end{equation}
The solution to this problem can be found by solving the associated
Bellman equation. In practical applications, it is almost always
intractable to solve the control problem to optimality.
Thus, algorithms based on approximations and heuristics are employed.
There are several proposed approximations in the
literature\todo{Mention a few?}, although most of the domain knowledge is kept within
respective commercial actors.
Many of these algorithms are justified solely on practical grounds, or
from asymptotic results.\todo{References}
In this paper, we will consider three\todo{double-check} approximations,
and look at their performance on a one-product system.
We show see that the optimal controller is more robust
miss-specification of the model uncertainty in the system.
Hence, the use of approximations puts more pressure on the
demand forecasting process.

We have organised the paper as follows:\todo{Write this out}
\begin{itemize}
\item Section 2: Mathematical formulation of the problem, an algorithm
  to solve it based on Bellman's equation, and an example with pricing rules.
\item Section 3: Discussion of some approximations (list them), and
  comparison with the bellman controller in the example
\item Section 4: What happens when the model is not correct?
\item (MAYBE) Section 5: Risk-averse. Formulation, impact
  (exponential, power, quadratic)
\item Section 6: Conclusion
\end{itemize}

\todo[inline]{Write about risk-averse formulations if we want to include it?}

\biblio
\end{document}

%%% Local Variables:
%%% mode: latex
%%% TeX-master: t
%%% End:
