\documentclass[main.tex]{subfiles}

\begin{document}

\abstract{
  In this article, we investigate  Approximate Dynamic Programming (ADP)
  algorithms for a pricing problem in retail.
  The pricing problem is formulated as a stochastic optimal
  control problem, where the optimal policy can be found by solving
  the associated Bellman equation.
  For realistic retail applications, modelling the problem and solving
  it to optimality becomes impractical and intractable.
  Thus practitioners make simplifying assumptions and
  design suboptimal policies, but a thorough investigation
  of the relative performance of these policies is lacking.
  A challenge in ADP is to better understand
  such trade-offs, to which this article contributes.
  We simulate the performance of a family of algorithms on a
  one-product system, and find that in
  particular situations, the popular, but approximate, Certainty
  Equivalent Control policy
  can perform better than an optimal, expected value maximising policy more than
  \emph{half of the time}.
  However, this approximate algorithm is less robust to model misspecification, and thus
  requires that the retailer spends more resources on state estimation.
}
\end{document}

%%% Local Variables:
%%% mode: latex
%%% TeX-master: t
%%% End:
