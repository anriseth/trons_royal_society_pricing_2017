\documentclass[main.tex]{subfiles}

\begin{document}

\abstract{
  In this paper, we investigate different approximate dynamic programming
  algorithms to a pricing problem for retail.
  The pricing problem is formulated as a stochastic optimal
  control problem, where the optimal policy can be found by solving
  the associated Bellman equation.
  For realistic retail applications, modelling the problem and solving
  the it to optimality becomes impractical and intractable.
  Thus practitioners make simplifying assumptions, but good investigations
  of their relative performance is lacking.
  The challenge with approximate dynamic programming is to understand
  the such trade-offs, which this paper contributes to.
  We simulate the performance on a one-product system, and find that
  certain situations, the popular certainty equivalent control policy
  can perform better than expected value maximising policy more than
  half of the time.
  However, this algorithm is less robust to model uncertainty, which
  requires the retailer to spend more resources on system estimation.
}
\end{document}

%%% Local Variables:
%%% mode: latex
%%% TeX-master: t
%%% End:
