\documentclass[main.tex]{subfiles}

\begin{document}
\pgfmathdeclarefunction{gauss}{3}{%
  \pgfmathparse{1/(#3*sqrt(2*pi))*exp(-((#1-#2)^2)/(2*#3^2))}%
}

\pgfmathdeclarefunction{gamma}{1}{%
  \pgfmathparse{2.506628274631*sqrt(1/#1)+ 0.20888568*(1/#1)^(1.5)+
    0.00870357*(1/#1)^(2.5)- (174.2106599*(1/#1)^(3.5))/25920-
    (715.6423511*(1/#1)^(4.5))/1244160)*exp((-ln(1/#1)-1)*#1)}%
}

\pgfmathdeclarefunction{gammapdf}{3}{%
  \pgfmathparse{1/(#3^#2)*1/(gamma(#2))*#1^(#2-1)*exp(-#1/#3)}%
}

\listoftodos

\section{Model uncertainty, or miss-specification}\label{sec:markdown_miss_specification}
It is difficult to correctly model the true underlying disturbance
$W^\dagger$, and the performance of the policy function should be
tested to different types of model miss-specification.
Say the policy functions are calculated based on
a the system disturbances $W_t\sim \frac{1}{2}+Beta(\mu,\mu)$, where
$\mu$ is chosen such that $\mathbb E[W]=1$ and $\mbox{std}[W]=\gamma$,
and consider two cases where the true disturbance is different:
\begin{enumerate}
\item Mean miss-specification, $W^\dagger = W-0.05$.
\item Standard deviation miss-specification, $\gamma^\dagger = 2\gamma$.
\end{enumerate}
We simulate the system for both of these cases 1000 times each,
and check the performance of the optimal Bellman policy and the
Certainty Equivalent Control policy. The results are shown in
\Cref{fig:markdown_bellman_mpc_model_misspecification}.
The relative performance of the CEC is drastically reduced when the
assumed mean is wrong, i.e.\ when a poor quality forecast is taken as
the truth.
\begin{figure}[htbp]
  \centering
  \begin{subfigure}[b]{0.5\textwidth}
    \begin{tikzpicture}[trim axis left]
      \begin{axis}[
        width=\textwidth,
        legend pos=north west,
        xlabel=$P^{\alpha^B}$,
        ylabel=Count,
        title={Mean miss-specification},
        xmin=0.52, xmax=0.71,
        ymax=150,
        ]
        \addplot[blue,hist={bins=40}] table [y index = 0,col sep=comma]
        {./data/markdown_bellman_mpc_model_misspecification.csv};
        \addlegendentry{$\mathbb E[W^\dagger]=0.95$};
      \end{axis}
    \end{tikzpicture}
  \end{subfigure}\hfill
  \begin{subfigure}[b]{0.5\textwidth}
    \begin{tikzpicture}
      \begin{axis}[
        width=\textwidth,
        legend pos=north west,
        xlabel=$P^{\alpha^B}$,
        title={Deviation miss-specification},
        xmin=0.52, xmax=0.71,
        ymax=150
        ]
        \addplot[blue,hist={bins=40}] table [y index = 1,col sep=comma]
        {./data/markdown_bellman_mpc_model_misspecification.csv};
        \addlegendentry{$\mbox{std}[W^\dagger]=0.1$};
      \end{axis}
    \end{tikzpicture}
  \end{subfigure}\\[1em]
  \begin{subfigure}[b]{0.5\textwidth}
    \begin{tikzpicture}[trim axis left]
      \begin{axis}[
        width=\textwidth,
        %         legend style={at={(0.5,0.97)}, anchor=north},
        xlabel=$P^{\alpha^B}-P^{\alpha^C}$,
        ylabel=Count,
        xmin=-0.007, xmax=0.025,
        ymax=500
        ]
        \addplot[blue,hist={bins=40}] table [y index = 4,col sep=comma]
        {./data/markdown_bellman_mpc_model_misspecification.csv};
        \addlegendentry{$\mathbb E[W^\dagger]=0.95$};
      \end{axis}
    \end{tikzpicture}
  \end{subfigure}\hfill
  \begin{subfigure}[b]{0.5\textwidth}
    \begin{tikzpicture}
      \begin{axis}[
        width=\textwidth,
        %         legend style={at={(0.5,0.97)}, anchor=north},
        xlabel=$P^{\alpha^B}-P^{\alpha^C}$,
        xmin=-0.007, xmax=0.025,
        ymax=500
        ]
        \addplot[blue,hist={bins=40}] table [y index = 5,col sep=comma]
        {./data/markdown_bellman_mpc_model_misspecification.csv};
        \addlegendentry{$\mbox{std}[W^\dagger]=0.1$};
      \end{axis}
    \end{tikzpicture}
  \end{subfigure}
  \caption{Performance of the Bellman controller and Certainty
  Equivalent Control policy under model miss-specification of
  $W$. Compare with \Cref{fig:markdown_bellman_mpc_5_20_100}.
  The policy functions are calculated under the assumption that
  the system disturbance is a translated Beta distribution with mean 1
  and standard deviation $5\times 10^{-2}$,
  instead
  of the true disturbance~$W^\dagger$
  The CEC policy performance is very dependent on mean miss-specification,
  as expected.
}\label{fig:markdown_bellman_mpc_model_misspecification}
\todo[inline]{Run simulations with beta distributions instead of normals}
\end{figure}

\subsection{Higher-order model uncertainty}
\todo[inline]{Redo with weird betas instead of chi^2_5?}
The model approximation of the two first moments of the underlying
disturbance is likely to be more accurate than for higher moments.
It can therefore be interesting to see how the different
policies deal with model uncertainty of higher moments.
We consider affine transformations of a $\chi^2_5$ distribution with five degrees of
freedom, that have the \emph{same mean and standard deviation as the model
distribution $W$}. A comparison of the probability density
functions can be seen in \Cref{fig:chisq_transformed}.
\begin{figure}[htbp]
  \centering
  \begin{subfigure}[b]{0.5\textwidth}
    \begin{tikzpicture}
      \begin{axis}[
        width=\textwidth,
        no markers, domain=0.9:1.25, samples=200,
        xmin=0.75, xmax=1.25,
        xlabel=$w$, ylabel=Density,
        legend cell align=left,
        title={Positive skewness}
        ]
        \addplot+[thick] {gammapdf(max((x-0.921)/0.0158,
        0),2.5,2)/0.0158};
        \addlegendentry{$W_1^{\dagger}$};
        \addlegendentry{$W$};
        \addplot+[thick,dashed,domain=0.75:1.25] {gauss(x,1.0,0.05)};
      \end{axis}
    \end{tikzpicture}
  \end{subfigure}%
  \begin{subfigure}[b]{0.5\textwidth}
    \begin{tikzpicture}
      \begin{axis}[
        width=\textwidth,
        no markers, domain=0.75:1.1, samples=200,
        xmin=0.75, xmax=1.25,
        xlabel=$w$, ylabel=Density,
        legend cell align=left,
        title={Negative skewness}
        ]
        \addplot+[thick] {gammapdf(max((1.079-x)/0.0158,
        0),2.5,2)/0.0158};
        \addlegendentry{$W_2^{\dagger}$};
        \addlegendentry{$W$};
        \addplot+[thick,dashed,domain=0.75:1.25] {gauss(x,1.0,0.05)};
      \end{axis}
    \end{tikzpicture}
  \end{subfigure}
  \caption{Probability density comparisons between
  the model disturbance $W$ and a true underlying
  model. A larger positive tail on the left ($W_1^\dagger$) and
  a larger negative tail on the right ($W_2^\dagger$).
}\label{fig:chisq_transformed}
\end{figure}
We again simulate the system with the Bellman and CEC policies 1000
times, and compare the outomes. The results are shown in
\cref{fig:markdown_bellman_mpc_chi2}. We see that in the two cases considered,
the relative performance of the CEC controller does not change much
compared to the case where the assumed uncertainty $W$ is the same as
the true uncertainty $W^\dagger$.

\begin{figure}[htbp]
  \centering
  \begin{subfigure}[b]{0.5\textwidth}
    \begin{tikzpicture}[trim axis left]
      \begin{axis}[
        width=\textwidth,
        legend pos=north west,
        xlabel=$P^{\alpha^B}$,
        ylabel=Count,
        title={Skewness under-specification},
        xmin=0.58, xmax=0.69,
        ymax=150,
        ]
        \addplot[blue,hist={bins=40}] table [y index = 2,col sep=comma]
        {./data/markdown_bellman_mpc_model_misspecification.csv};
        \addlegendentry{$W_1^\dagger$};
      \end{axis}
    \end{tikzpicture}
  \end{subfigure}\hfill
  \begin{subfigure}[b]{0.5\textwidth}
    \begin{tikzpicture}
      \begin{axis}[
        width=\textwidth,
        legend pos=north west,
        xlabel=$P^{\alpha^B}$,
        title={Skewness over-specification},
        xmin=0.58, xmax=0.69,
        ymax=150
        ]
        \addplot[blue,hist={bins=40}] table [y index = 3,col sep=comma]
        {./data/markdown_bellman_mpc_model_misspecification.csv};
        \addlegendentry{$W_2^\dagger$};
      \end{axis}
    \end{tikzpicture}
  \end{subfigure}\\[1em]
  \begin{subfigure}[b]{0.5\textwidth}
    \begin{tikzpicture}[trim axis left]
      \begin{axis}[
        width=\textwidth,
        %         legend style={at={(0.5,0.97)}, anchor=north},
        xlabel=$P^{\alpha^B}-P^{\alpha^C}$,
        ylabel=Count,
        xmin=-0.006, xmax=0.022,
        ymax=600
        ]
        \addplot[blue,hist={bins=40}] table [y index = 6,col sep=comma]
        {./data/markdown_bellman_mpc_model_misspecification.csv};
        \addlegendentry{$W_1^\dagger$};
      \end{axis}
    \end{tikzpicture}
  \end{subfigure}\hfill
  \begin{subfigure}[b]{0.5\textwidth}
    \begin{tikzpicture}
      \begin{axis}[
        width=\textwidth,
        %         legend style={at={(0.5,0.97)}, anchor=north},
        xlabel=$P^{\alpha^B}-P^{\alpha^C}$,
        xmin=-0.006, xmax=0.022,
        ymax=600
        ]
        \addplot[blue,hist={bins=40}] table [y index = 7,col sep=comma]
        {./data/markdown_bellman_mpc_model_misspecification.csv};
        \addlegendentry{$W_2^\dagger$};
      \end{axis}
    \end{tikzpicture}
  \end{subfigure}
  \caption{Performance of Bellman and CEC when the true system disturbances
  are transformations $W_1^\dagger,W_2^\dagger$ of
  a $\chi_5^2$ distribution. Compare to
  \Cref{fig:markdown_bellman_values,fig:markdown_bellman_mpc_1_20_100}.
  The relative performance of CEC compared to Bellman does not
  change much,
  although we see a slight deterioration for the under-estimation of
  skewness (left) and a slight improvement when the skewness has
  been over-estimated (right).
}\label{fig:markdown_bellman_mpc_chi2}
\todo[inline]{Rethink the ``skewness''-descriptions}
\end{figure}

\biblio
\end{document}
%%% Local Variables:
%%% mode: latex
%%% TeX-master: t
%%% TeX-command-extra-options: "-shell-escape"
%%% End:
