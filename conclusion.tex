\documentclass[main.tex]{subfiles}

\begin{document}

\listoftodos

\section{Conclusion}\label{sec:conclusion}
In this article we have looked at a mathematical formulation of a
retail pricing problem for profit maximisation, and have investigated the
performance of two algorithms that balance practicality
with degree of suboptimality. The motivation is to better understand
how well the suboptimal policies approximate the returns from the
optimal policy.
Pricing problems are often formulated as an expected value
maximisation, but different algorithms may induce different
distributions of the profits.
So even though the expected values of suboptimal policies are not better
than the optimal policy, they may have a higher profit for some
realisations of the underlying probability distribution.
We found in \Cref{sec:suboptimal_approximations} that there are a
large number of
reasonable system parameters for which
the suboptimal Certainty Equivalent Control policy resulted in a higher profit
than the optimal policy in more than half of the realisations.
However, in the remaining realisations, the CEC policy resulted in
much smaller profits. We interpret these results as an indication that
the suboptimal policy is more risk-seeking than the optimal policy.
The results in this article underscore the importance of looking at
the impact of
different suboptimal algorithms have on the distribution of
an objective, and not only the impact of
marginalised statistics such as the expected value.

The model problem in this article is fairly simple, and
we propose two specific lines for future research that
take this analysis closer to practical models.
First, to investigate multi-product problems where the demand and
availability of one product depends on the other products.
Second, to introduce a state dependence in the uncertainty in the
system, that is, to allow $W_{t+1}$ to depend explicitly on the values
of $S_t$ and $\alpha_t$.
We hope that these two extensions will lead to a better understanding of
whether the effects seen in this article will be stronger or diluted
in real-life problems.

\biblio
\end{document}

%%% Local Variables:
%%% mode: latex
%%% TeX-master: t
%%% TeX-command-extra-options: "-shell-escape"
%%% End:
