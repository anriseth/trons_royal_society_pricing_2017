\documentclass[main.tex]{subfiles}

\begin{document}

\listoftodos

\section{Conclusion}\label{sec:conclusion}
In this paper we have looked at a mathematical formulation of a
retail pricing problem for profit maximisation, and investigated the
performance of different algorithms that balance practicality
with degree of suboptimality. The motivation is to better understand
how well the suboptimal policies approximate the returns from the
optimal policy.
Pricing problems are often formulated as an expected value
maximisation, but different algorithms may create different
distributions of the realised profits.
So even though the expected value of suboptimal policies are no better
than the optimal policy, they may have a higher profit for some
realisations of the underlying uncertainty.
We found in \Cref{sec:cec_comparison_example} that there are
reasonable system parameters for which
the suboptimal Certainty Equivalent Control policy performs better
than the optimal one in more
than half of the realisations.
This underscores the importance of looking at what impact
different suboptimal algorithms have on the distribution of
an objective, and not only the impact of
marginalised statistics like the expected value.
\todo{Mention model uncertainty?}

The model problem in this paper is fairly simple, and
we propose two specific lines of research that
take this analysis closer to practical models.
First, to investigate multi-product problems where the demand and
availability of one product depend on the other products.
Second, to introduce a state dependence on the uncertainty in the
system, that is to allow $W_{t+1}$ to depend explicitly on the values
of $S_t$ and $\alpha_t$.
We hope that these two extensions can get a better understanding of
whether the effects seen in this paper will be stronger or diluted
in real-life problems.

\biblio
\end{document}

%%% Local Variables:
%%% mode: latex
%%% TeX-master: t
%%% TeX-command-extra-options: "-shell-escape"
%%% End:
