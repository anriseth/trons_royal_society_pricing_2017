\documentclass[main.tex]{subfiles}

% This code shows which label has been changed when latex complains
%\makeatletter
% \def\@testdef #1#2#3{%
%   \def\reserved@a{#3}\expandafter \ifx \csname #1@#2\endcsname
%  \reserved@a  \else
% \typeout{^^Jlabel #2 changed:^^J%
% \meaning\reserved@a^^J%
% \expandafter\meaning\csname #1@#2\endcsname^^J}%
% \@tempswatrue \fi}
%\makeatother

\begin{document}

\listoftodos

\section{The optimal control problem}
The overarching goal in the paper is to consider solutions to a retail
pricing problem.
\begin{mydef}[Pricing problem, informal]
  Given a stock of a product and a future terminal time,
  dynamically set the price
  of the product in order to maximise revenue and minimise the cost of
  unsold stock.
\end{mydef}
In this section, we define the optimal control pricing problem,
describe the optimality conditions given by the Bellman equation and
solve it numerically for an example system.

Consider a system over discrete time points $t=0,\dots,T$, with state
$S$ and policy process $\alpha$ that takes values in an
interval $A\subset\mathbb R_+$.
The state $S_t$ denotes the stock of product at time $t$, and
$\alpha_t$ the price for the product in the time period from $t$ to
$t+1$. We model the sales of the product according to a demand
function $q:A\to\mathbb R_+$ that is bounded, continuous and decreasing.
Randomness in the system to capture exogenous influences on demand, is
modelled in a multiplicative fashion by a Markovian stochastic process
$(W)_t$ taking non-negative values. See for example
\citet[Ch.~7]{talluri2006theory} for a discussion of
popular demand models and modelling of uncertainty.

For a given pricing process $\alpha$, the dynamics of
the system evolves from some initial state $S_0=s>0$, according to the
recursion
\begin{equation}\label{eq:stock_dynamics}
  S_{t+1}^\alpha=S_t^\alpha-\min(q(\alpha_t)W_{t+1},S_t^\alpha),\qquad t=0,\dots,T-1.
\end{equation}
The function $Q(s,a,w)=\min(q(a)w,s)$ denotes the unit sales per
period at price $a$,
starting with stock $s$, and exogenous influences
characterised by $w$.

The revenue accrued over period $t\to t+1$ is $\alpha_tQ(S_t^\alpha,\alpha_t,W_{t+1})$.
A cost of remaining stock at time $T$ is modelled by a cost per unit
stock $C\geq 0$.\footnote{In some situations, unsold items at time $T$
  may be sold at some ``salvage price'', in which case we could allow
  $C<0$}
Let $\mathcal A$ denote the set of feasible processes that take values
in $A$.
Define the value of having stock $s$ at time $t\leq T$
by the \emph{value function}
\begin{align}\label{eq:value_function_def}
  v(t,s)&=\max_{\alpha\in\mathcal A} J(t,s,\alpha),\quad\text{where}\\
  J(t,s,\alpha)&=
                 \mathbb E_{W}\left[ \sum_{\tau=t}^{T-1}
                 \alpha_tQ(S_\tau^\alpha,\alpha_\tau,W_{\tau+1})
                 - CS_T^\alpha \mid S_t^\alpha = s
                 \right].
                 \label{eq:value_function_def2}
\end{align}
This leads us to the following mathematical formulation of the pricing
problem:
\begin{mydef}[Pricing problem]
  Given an initial stock $s>0$ and a cost per unit unsold stock $C\geq
  0$, find $\alpha\in\mathcal A$ such that
  \begin{equation}
    J(0,s,\alpha) = v(0,s).
  \end{equation}
\end{mydef}
We choose to maximise the expected profit over the period, which
assumes a risk-neutral decision-maker. However, it is still important
to understand the distribution of profits for a given pricing policy
$\alpha$. Therefore, we look at the full distribution when
we investigate the performance of algorithms in the rest of the paper.
This stochastic optimal control problem can be solved by
considering the optimality conditions that arise from the Dynamic
Programming principle, also known as the Bellman equation.

\subsection{The Bellman equation}
We assume the optimal policy $\alpha\in\mathcal A$ is Markovian,\todo{
  do I need to justify that the optimal policy is Markovian?}
meaning that
there exists a function $a$ such that $\alpha_t(\omega) =
a(t,S_t^\alpha(\omega))$ for each $\omega$ in the underlying probability space.
Then,
an approach to finding the value function above is to use the Dynamic
Programming principle, which states that $v$ can be defined
recursively in the following way:
\begin{align}\label{eq:dynamic_programming_discrete}
  v(t,s)&=\max_{a\in A}\mathbb E_{W}\left[
          aQ(s,a,W_{t+1})
          +v(t+1,s-Q(s,a,W_{t+1}))\right].
\end{align}
Thus, the value function is the solution to the backwards-in-time
recursive relation~\eqref{eq:dynamic_programming_discrete} with
terminal value $v(T,s)=-Cs$, and the optimal policy
function $a(t,s)$ is given by the argmax for each $t,s$.
The recursion is called the \emph{Bellman equation}, and a discussion
of its validity can be found for example in \citet{bertsekas2005dynamic}.

We implement the following algorithm to solve the optimal control
problem, using the Bellman equation:\todo{Decorate with some algorithm environment?}
\begin{enumerate}
\item Create grid $s_1,\dots,s_K$, and arrays $v^K\in\mathbb R^{K\times(T+1)}$,
  $\alpha^K\in\mathbb R^{K\times T}$
\item Set $Iv^K(s)=-Cs$
\item Set $v^K[i,T]=Iv^K(s_i)$ for $i=1,\dots, K$
\item For $t = T-1,\dots,0$
  \begin{enumerate}
  \item Set $\displaystyle v^K[i,t]=\max_{a\in A}\mathbb E_{W_{t+1}}\left[ aQ(s,a,W_{t+1})
      +Iv^K(s-Q(s,a,W_{t+1}))\right]$\\ for $i=1,\dots,K$.
  \item Set $\alpha^K[i,t]$ to the maximiser above
  \item Set $Iv^K(s) = Interpolate(s, {(s_i)}_i,{(v^K[i,t])}_i)$
  \end{enumerate}
\item Return $v^K,\alpha^K$
\end{enumerate}

\subsection{Example system}\label{sec:bellman_example_markdown}
For the rest of the paper, we consider a given non-dimensionalised
system, in such a way that stock levels lie in $[0,1]$ and the price
interval $A=[0,1]$.
Let the demand function be given by
\begin{equation}
  q(a)=\frac{1}{3}e^{2-3a}.
\end{equation}
We assume the exogenous disturbance process is a sequence of
i.i.d.~shifted Beta-distributed random variables with mean 1 and variance
$\gamma^2$. That is, we set $W_t\sim \frac{1}{2}+X$, where
$X\sim Beta(\mu,\nu)$, and
$\mu=\nu=\frac{1}{8\gamma^2}-\frac{1}{2}$.

Set $\gamma = 5\times 10^{-2}$, $C=1$ and $T=3$.
The solution to the Bellman equation, and the corresponding optimal
pricing policy is shown in \Cref{fig:markdown_bellman}.
\begin{figure}[htbp]
  \centering
  \begin{subfigure}[b]{0.5\textwidth}
    \begin{tikzpicture}[scale=0.8]
      \begin{axis}[
        xlabel={$s$},
        ylabel={$v(t,s)$},
        title={Value function},
        legend cell align=left,
        legend pos=north west
        ]
        \addplot+[mark=none] table[x index = 0,y index = 1,col sep=comma]
        {./data/markdown_bellman_det_val_policy.csv};
        \addlegendentry{$t=0$};
        \addplot+[mark=none] table[x index = 0,y index = 2,col sep=comma]
        {./data/markdown_bellman_det_val_policy.csv};
        \addlegendentry{$t=1$};
        \addplot+[mark=none] table[x index = 0,y index = 3,col sep=comma]
        {./data/markdown_bellman_det_val_policy.csv};
        \addlegendentry{$t=2$};
      \end{axis}
    \end{tikzpicture}
    % \includegraphics[width=\textwidth]{./img/markdown_value_bellman}
  \end{subfigure}%
  \begin{subfigure}[b]{0.5\textwidth}
    \begin{tikzpicture}[scale=0.8]
      \begin{axis}[
        xlabel={$s$},
        ylabel={$a(t,s)$},
        title={Policy function},
        legend cell align=left,
        ]
        \addplot+[mark=none] table[x index = 0,y index = 5,col sep=comma]
        {./data/markdown_bellman_det_val_policy.csv};
        \addlegendentry{$t=0$};
        \addplot+[mark=none] table[x index = 0,y index = 6,col sep=comma]
        {./data/markdown_bellman_det_val_policy.csv};
        \addlegendentry{$t=1$};
        \addplot+[mark=none] table[x index = 0,y index = 7,col sep=comma]
        {./data/markdown_bellman_det_val_policy.csv};
        \addlegendentry{$t=2$};
      \end{axis}
    \end{tikzpicture}
    % \includegraphics[width=\textwidth]{./img/markdown_controls_bellman}
  \end{subfigure}
  \caption{The value function and corresponding
    optimal control function for the pricing problem.
    The kink in the value function
    correspond to when the pricing policy hits the upper bound 1.
  }\label{fig:markdown_bellman}
\end{figure}
Let us now investigate the behaviour of the optimal pricing policy
$\alpha$ and
the outcome of following this policy.
Define a random variable $P^\alpha$, which for each realisation
represents the total profit,
\begin{equation}
  P^\alpha = \sum_{t=0}^T\alpha_tQ(S_t^\alpha,\alpha_t,W_{t+1}) - CS_T^\alpha.
\end{equation}
If $a(t,s)$ is the function found when solving the Bellman equation,
then $\alpha$ is a stochastic process defined for each event $\omega$
from the underlying probability space, with
$\alpha_t(\omega)=a(t,S_t^\alpha(\omega))$.
By sampling from the stochastic process $W=(W_1,\dots,W_T)$, we can
estimate the random variables $\alpha_t$ and $P^\alpha$.
The plots in \Cref{fig:bellman_simulation} show the results of
simulating the system 1000 times.

\begin{figure}[htbp]
  \centering
  \begin{subfigure}[t]{0.5\textwidth}
    \begin{tikzpicture}[scale=0.8]
      \begin{axis}[
        ylabel={Price},
        xlabel={\phantom{$P^\alpha$}},
        title={$\alpha_t$, simulated values},
        boxplot/draw direction=y,
        xtick={1,2,3},
        xticklabels={$t=0$, $t=1$, $t=2$},
        %ymin=0.6,ymax=0.75
        ]
        \addplot+[boxplot] table[y index=0,col sep=comma]
        {./data/markdown_bellman_det_policies.csv};
        \addplot+[boxplot] table[y index=1,col sep=comma]
        {./data/markdown_bellman_det_policies.csv};
        \addplot+[boxplot] table[y index=2,col sep=comma]
        {./data/markdown_bellman_det_policies.csv};
      \end{axis}
    \end{tikzpicture}
  \end{subfigure}%
  \begin{subfigure}[t]{0.5\textwidth}
    \begin{tikzpicture}[scale=0.8]
      \begin{axis}[
        xlabel=$P^\alpha$,
        ylabel=Count,
        title={Realised profit},
        legend cell align=left
        ]
        \addplot[blue,hist={data=x,bins=40}] table [y index = 0, col
        sep=comma]
        {./data/markdown_bellman_det_vals.csv};
      \end{axis}
    \end{tikzpicture}
  \end{subfigure}%
  % \begin{subfigure}[b]{0.5\textwidth}
  %   \begin{tikzpicture}[scale=0.8]
  %     \begin{axis}[
  %       title={$\alpha_t^C$, simulated values},
  %       boxplot/draw direction=y,
  %       xtick={1,2,3},
  %       xticklabels={$t=0$, $t=1$, $t=2$},
  %       ymin=0.6,ymax=0.75
  %       ]
  %       \addplot+[boxplot] table[y index=3,col sep=comma]
  %       {./data/markdown_bellman_det_policies.csv};
  %       \addplot+[boxplot] table[y index=4,col sep=comma]
  %       {./data/markdown_bellman_det_policies.csv};
  %       \addplot+[boxplot] table[y index=5,col sep=comma]
  %       {./data/markdown_bellman_det_policies.csv};
  %     \end{axis}
  %   \end{tikzpicture}
  % \end{subfigure}%
  \caption{Simulations of the pricing system, started at $S_0=1$ % TODO: subscript _0 causes issues with label re-run warnings
    and
    controlled by the optimal policy $\alpha$ seen in
    \Cref{fig:markdown_bellman}.
    The left figure shows a box plot that represents the distribution of
    realised prices, and the right is a histogram of the profit over
    the pricing period. As we see, the variance of the prices increase
    in time, reflecting the wider range of remaining stock at these times.
  }\label{fig:bellman_simulation}
\end{figure}

\todo[inline]{Do we need some sort of ``concluding'' remarks here?}

\biblio
\end{document}

%%% Local Variables:
%%% mode: latex
%%% TeX-master: t
%%% TeX-command-extra-options: "-shell-escape"
%%% End:
