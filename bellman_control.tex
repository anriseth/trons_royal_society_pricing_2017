\documentclass[main.tex]{subfiles}

\begin{document}

\listoftodos[Notes]

\section{The optimal control problem}
The overarching goal in the paper is to consider solutions to a retail
pricing problem.
\begin{mydef}[Pricing problem, informal]
  Given a stock of a product and a future terminal time,
  dynamically set the price
  of the product in order to maximise revenue and minimise the cost of
  unsold stock.
\end{mydef}
In this section, we define the optimal control pricing problem,
describe the optimality conditions given by the Bellman equation and
solve it numerically for an example system.

Consider a system over discrete time points $t=0,\dots,T$, with state
$S$ and policy process $\alpha$ that takes values in an
interval $A\subset\mathbb R_+$.
The state $S_t$ denotes the stock of product at time $t$, and
$\alpha_t$ the price for the product in the time period from $t$ to
$t+1$. We model the sales of the product according to a demand
function $Q:A\to\mathbb R_+$ that is bounded, continuous and decreasing.
Randomness in the system to capture exogenous influences on demand, is
modelled in a multiplicative fashion by a Markovian stochastic process
$(W)_t$ taking non-negative values. See for example
\citet[Ch.~7]{talluri2006theory} for a discussion of
popular demand models and modelling of uncertainty.

For a given pricing process $\alpha$, the dynamics of
the system evolves from some initial state $S_0=s>0$, according to the
recursion
\begin{equation}\label{eq:stock_dynamics}
  S_{t+1}^\alpha=S_t^\alpha-\min(Q(\alpha_t)W_{t+1},S_t^\alpha),\qquad t=0,\dots,T-1.
\end{equation}
The function $f(s,a,w)=\min(Q(a)w,s)$ denotes the unit sales per
period at price $a$,
starting with stock $s$, and exogenous influences
characterised by $w$.

The revenue accrued over period $t\to t+1$ is $\alpha_t(S_t^\alpha-S_{t+1}^\alpha)$.
A cost of remaining stock at time $T$ is modelled by a cost per unit
stock $C\geq 0$.\footnote{In some situations, unsold items at time $T$
  may be sold at some ``salvage price'', in which case we could allow
  $C<0$}
Let $\mathcal A$ denote the set of feasible processes that take values
in $A$.
Define the value of having stock $s$ at time $t\leq T$
by the \emph{value function}
\begin{align}
  v(t,s)&=\max_{\alpha\in\mathcal A} J(t,s,\alpha),\quad\text{where}\\
  J(t,s,\alpha)&=
                 \mathbb E_{W}\left[ \sum_{\tau=t}^{T-1} \alpha_t\min(Q(\alpha_\tau)W_{\tau+1},S_\tau^\alpha) -
                 CS_T^\alpha \mid S_t^\alpha = s
                 \right].
\end{align}
This leads us to the following mathematical formulation of the pricing
problem:
\begin{mydef}
  Given an initial stock $s>0$ and a cost per unit unsold stock $C\geq
  0$, find $\alpha\in\mathcal A$ such that
  \begin{equation}
    J(0,s,\alpha) = v(0,s).
  \end{equation}
\end{mydef}

\subsection{The deterministic case}
First, let us assume the system is deterministic, with
$W\equiv 1$.
Then the pricing problem is a classic finite-dimensional optimisation
problem
\begin{equation}
  v(t,s)=\max_{\mathbf a\in A^{T-t}}\left\{\sum_{\tau=t}^{T-1}(\mathbf
    a_t+C)Q(\mathbf a_t)-Cs\right\},
  \quad \text{s.t.}\quad \sum_{\tau=t}^{T-1}Q(\mathbf a_t)\leq s.
\end{equation}
The optimal choice here is to choose $\mathbf a_t=a^*\in A$ for each
$t$, where $a^*$ solves
\begin{equation}
  \max_{a\in A} \left\{(a+C)Q(a)\right\},\quad\text{s.t.}\quad
  Q(a)\leq \frac{s}{t}.
\end{equation}

We solve the problem for two families of demand functions popular in
the literature, see e.g.~\citep[Ch.~7]{talluri2006theory}.
Let $\mathcal P_A$ be the projection operator onto the interval $A$.
If $Q$ is of the form $a\mapsto q_1-q_2a$, with $q_1,q_2> 0$, then
the optimal policy $\alpha_t=a(t,S_t^\alpha)$ for the deterministic
problem is given by the function
\begin{equation}
  a(t,s)=\mathcal P_A \left[ \max\left(
      \frac{q_1}{q_2}-\frac{s}{q_2(T-t)},\frac{1}{2}\left(\frac{q_1}{q_2}-C
      \right) \right) \right].
\end{equation}
In the case when $Q(a)=q_1e^{-q_2a}$, the policy function is
\begin{equation}
  a(t,s)=\mathcal P_A\left[
    \max\left( \frac{1}{q_2}\log\left( \frac{T-t}{q_1s}\right),
      \frac{1}{q_2}-C  \right)\right].
\end{equation}
These policy functions are suboptimal in the stochastic control
setting, and we wish to compare them and other methods to the optimal
policy, which can be found by solving the \emph{Bellman equation}.

\subsection{The Bellman equation}
We assume $\alpha\in\mathcal A$ is Markovian,\todo{
  do I need to justify that the optimal policy is Markovian?}
meaning that
there exists a function $a$ such that $\alpha_t(\omega) =
a(t,S_t^\alpha(\omega))$ for each $\omega$ in the underlying probability space.
Then,
an approach to find the value function above is to use the Dynamic
Programming principle, which states that $v$ can be defined
recursively in the following way:
\begin{align}\label{eq:dynamic_programming_discrete}
  v(t,s)&=\max_{a\in A}\mathbb E_{W}\left[
          af(s,a,W_{t+1})
          +v(t+1,s-f(s,a,W_{t+1}))\right].
\end{align}
Thus, the value function is the solution to the backwards-in-time
recursive relation~\eqref{eq:dynamic_programming_discrete} with
terminal value $v(T,s)=-Cs$, and the optimal policy
function $a(t,s)$ is given by the argmax for each $t,s$.
The recursion is called the \emph{Bellman equation}, and a discussion
of its validity can be found for example in \citet{bertsekas2005dynamic}.

We implement the following algorithm to solve the optimal control
problem, using the Bellman equation:\todo{Decorate with some algorithm environment?}
\begin{enumerate}
\item Create grid $s_1,\dots,s_K$, and arrays $v^K\in\mathbb R^{K\times(T+1)}$,
  $\alpha^K\in\mathbb R^{K\times T}$
\item Set $Iv^K(s)=-Cs$
\item Set $v^K[i,T]=Iv^K(s_i)$ for $i=1,\dots, K$
\item For $t = T-1,\dots,0$
  \begin{enumerate}
  \item Set $\displaystyle v^K[i,t]=\max_{a\in A}\mathbb E_{W_{t+1}}\left[ af(s,a,W_{t+1})
      +Iv^K(s-f(s,a,W_{t+1}))\right]$\\ for $i=1,\dots,K$.
  \item Set $\alpha^K[i,t]$ to the maximiser above
  \item Set $Iv^K(s) = Interpolate(s, {(s_i)}_i,{(v^K[i,t])}_i)$
  \end{enumerate}
\item Return $v^K,\alpha^K$
\end{enumerate}

\subsection{Example system}\label{sec:bellman_example_markdown}
For the rest of the paper, we consider a given non-dimensionalised
system, in such a way that stock levels lie in $[0,1]$ and the price
interval $A=[0,1]$.
Let the demand function be given by
\begin{equation}
  Q(a)=\frac{1}{3}e^{2-3a}.
\end{equation}
We assume the exogenous disturbance process is a sequence of
i.i.d.~shifted Beta-distributed random variables with mean 1 and variance
$\gamma^2$, where $0<\gamma<<1$.\footnote{Set $W_t\sim 0.5+X$, where
  $X\sim Beta(\alpha,\beta)$, and
  $\alpha=\beta=\frac{1}{8\gamma^2}-\frac{1}{2}$.}

Set $\gamma = 5\times 10^{-2}$, $C=1$ and $T=3$.
The solution to the Bellman equation, and the corresponding optimal
pricing policy is shown in \Cref{fig:markdown_bellman}.
Note\todo{move to plot of simulated prices} that for the open loop,
deterministic version of the control problem, with
$W\equiv 1$, the optimal policy is the constant function
$\alpha\equiv \frac{2}{3}$.
\begin{figure}[ht]
  \centering
  \begin{subfigure}[b]{0.5\textwidth}
    \includegraphics[width=\textwidth]{./img/markdown_value_bellman}
  \end{subfigure}%
  \begin{subfigure}[b]{0.5\textwidth}
    \includegraphics[width=\textwidth]{./img/markdown_controls_bellman}
  \end{subfigure}
  \caption{The value function and corresponding
    optimal control function for the markdown problem.
    The kink in the value function near stock levels 0.2
    correspond to when the pricing policy hits the upper bound 1.
  }\label{fig:markdown_bellman}
  \todo[inline]{Redo with Beta-distribution instead of normal}
  \todo[inline]{Include $t=0$ curves as well}
\end{figure}

\todo[inline]{Compare against certainty equivalent controller}
\todo[inline]{Plot pricing curves from simulations: mean and quantiles}

\biblio
\end{document}

%%% Local Variables:
%%% mode: latex
%%% TeX-master: t
%%% TeX-command-extra-options: "-shell-escape"
%%% End:
