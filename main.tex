%%% Local Variables:
%%% mode: latex
%%% TeX-master: "main"
%%% TeX-command-extra-options: "-shell-escape"
%%% End:

\documentclass[a4paper,12pt]{article}
\pdfoutput=1

\usepackage{mathmacros} % My own macros
\usepackage{authblk}
\setcitestyle{square,numbers} % natbib
\setlength{\marginparwidth}{3.1cm} % For todonotes

\pgfplotsset{compat=1.13}
\usetikzlibrary{external}
\usetikzlibrary{pgfplots.statistics}
\tikzexternalize
\tikzsetexternalprefix{figures/}
\makeatletter
\renewcommand{\todo}[2][]{\tikzexternaldisable\@todo[#1]{#2}\tikzexternalenable}
\makeatother
% \usepackage[margin=2.5cm]{geometry}
\usepackage{breqn}
% \usepackage{todo}

\usepackage{subfiles}
\def\biblio{\bibliographystyle{abbrvnat}\bibliography{references}}

\title{Stochastic control strategies applied to a
  pricing problem in retail}
\author{Asbj{\o}rn Nilsen Riseth\thanks{
    This publication is based on work partially supported by the EPSRC
    Centre For Doctoral Training in Industrially Focused Mathematical
    Modelling (EP/L015803/1) in collaboration with a partner
    company.\\
    The author would like to thank his supervisors
    J.~N.~Dewynne and C.~L.~Farmer for their valuable suggestions.}\\
  {\footnotesize\texttt{riseth@maths.ox.ac.uk}}}
\affil{Mathematical Institute, University
  of Oxford, OX2 6GG.}

\date{\today}

\begin{document}
\maketitle
\listoftodos

\def\biblio{}
\def\listoftodos{}

\subfile{abstract}

\vspace{1em}\noindent
\textbf{Key words:} Revenue management, dynamic pricing, stochastic
optimal control,
approximate dynamic programming.

\section{Feedback from Sam Cohen}
Key points he had issues with:
\begin{itemize}
\item I am only considering one model (he did not find the appendix
  bit useful, or at least I would have to do it over $T=3$ periods
  there as well)
  \begin{itemize}
  \item This seems to be a criticism for several things: choosing only one
    demand-model (and not justifying it), one set of parameters for
    this model, and one type of model disturbance
  \end{itemize}
\item My Figure caption of ``better 50\%'' of the time sounded like I
  am saying ``CEC is a better method''. He wanted me to conclude (as
  we have done), that it means higher risk / fatter lower tail
\item It seems like he felt the conclusions were ``obvious'',
  i.e.~that such a ``terrible'' approximation as CEC will obviously
  create lower mean and larger variance
\item Model uncertainty: Sam said that I have just shown things for
  very particular models, but could not extrapolate to statements like ``CEC
  is unstable to mean misspecification''
\item Comment on what types of events cause the ``CEC'' better
  outcomes [do the experiment first]
\end{itemize}

For future work, his opinion was that I should also compare something in between
CEC and full Bellman, as they are two extremes. Especially when the
state and decision space dimensions become larger (e.g. let all prices
be defined in terms of three factors, instead of having each product
have their own control function [I think])
This should give a table with relevant metrics after simulation (mean,
variance, VaR, time to solve)

Reading list from Sam:
\begin{itemize}
\item Almgren and Chriss, 2000: Optimal Execution of Portfolio
  Transactions
\item Obizhaeva+Wang: Optimal Trading Strategy
and
Supply/Demand Dynamics (\textbf{I think?})
\end{itemize}

Things I think can improve the paper (and make it more precise)
\begin{itemize}
\item Only compare Bellman and CEC
\item Emphasise that CEC is what is done in the revenue literature
\item Emphasise that the models we use are the ones from literature
  [or maybe use the GBM model]
\item Present the distribution plot for one demand model, with
  two-three different parameters
\item Show mean, variance, VaR (e.g. 95\%) comparison for three demand
  models, and three parameter combinations
\item Either scrap model uncertainty, or justify better why I just
  shift the mean but keep the model the same
\end{itemize}


\subfile{introduction}
\subfile{bellman_control}
\subfile{control_comparison}
\subfile{model_uncertainty}
% \subfile{expected_utility}
\subfile{conclusion}

% Bibliography:
\clearpage
\bibliographystyle{abbrvnat}
\bibliography{references}


%\appendix
\subfile{appendix_policydiff}


\end{document}
